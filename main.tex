\documentclass[twocolumn]{article}
\usepackage{amssymb}
\usepackage[table,xcdraw]{xcolor}
\usepackage{amsmath}
\usepackage[utf8]{inputenc}
\usepackage[spanish]{babel}
\usepackage[sc]{mathpazo} 
\usepackage{ upgreek }
\linespread{1.05}
\usepackage{microtype}
\usepackage[hang, small,labelfont=bf,up,textfont=it,up]{caption}
\usepackage{lettrine}
\usepackage{graphics}
\usepackage[hmarginratio=1:1,top=20mm,bottom=20mm,right=10mm,left=10mm,columnsep=10pt]{geometry}
\usepackage{graphicx} 
\usepackage{booktabs}
\usepackage{float} 
\usepackage{subfigure}
\usepackage{paralist} 
\usepackage{hyperref} 
\usepackage{abstract}
\renewcommand{\abstractnamefont}{\normalfont\bfseries}
\renewcommand{\abstracttextfont}{\normalfont\small\itshape}
\usepackage{titlesec}
\renewcommand\thesection{\Roman{section}} 
\renewcommand\thesubsection{\Roman{subsection}} 
\titleformat{\section}[block]{\large\scshape\centering}{\thesection.}{1em}{} 
\titleformat{\subsection}[block]{\large}{\thesubsection.}{1em}{}

\usepackage{fancyhdr} 
\usepackage{wrapfig}
\usepackage{blindtext}




%\fancyfoot[RO,LE]{\thepage} \newenvironment{Figure}
 % {\par\medskip\noindent\minipage{\linewidth}}
%  {\endminipage\par\medskip}



\title{\vspace{-18mm}\fontsize{18pt}{20pt}\selectfont\textbf{Intensidad del campo magnético de un imán de neodímio a distintas temperaturas}} 


\author{
\large
\textsc{Noemí de la peña, Benjamín Opazo, Martina Contreras, Jeremías Martínez}\\[2mm]
\large Estudiantes de Cs. Físicas, Universidad de Concepción. 
%\vspace{2mm}
}
\date{}




\begin{document}

\pagestyle{fancy}
\lhead[]{Laboratorio I}
\chead[]{}
\rhead[]{Experimento}

\maketitle

%\headrule


\begin{abstract}
    En el siguiente informe, se muestran los resultados del estúdio del campo magnético de un imán de neodimio y su reacción al modificar la temperatura de este metal. Mediante un análisis cuantitativo de las variables de interés examinaremos la situación mediante la recolección empírica de datos del campo obtenidos por medición directa utilizando el magnétometro incluido como herramienta de phyphox\cite{ref2}, al estudiar una porción pequeña de dicho material a distintas temperaturas se puede inferir de los datos registrados, que a medida que la temperatura incrementa, el campo magnético disminuye, y de forma análoga al disminuir la temperatura en el imán, su intensidad magnética aumenta.
\end{abstract}
\section{Introducción}
El imán y sus propiedades magnéticas fueron descubiertas por Tales de mineto, en los años 600 A.C.
El descubrimiento de imán dividido en un antes y después la historia de la humanidad, dado como origen a 2 ramas físicas muy importantes en la actualidad , como: El magnetismo y el electromagnetismo.

En este informe presentaremos una variedad de datos , relacionados con la variación del campo magnético a distintas temperaturas.
Donde primero presentaremos los objetivos de nuestra investigación. Luego definir algunos conceptos importantes, para entender con claridad lo sucedido en el laboratorio. Después expondremos los materiales y procedimientos utilizados. 
Finalizaremos exponiendo los datos, tanto en tablas como en graficas. Y en consiguiente, diremos si se cumplieron los objetivos de nuestra investigación.
\section{Marco teórico}
\begin{itemize}
    \item \textbf{Momento magnético} es la fuerza magnética y la orientación de un imán u otro objeto que produce un campo magnético. Los ejemplos de objetos que tienen momentos magnéticos incluyen bucles de corriente eléctrica, imanes permanentes, partículas elementales, varias moléculas y muchos objetos astronómicos.
    \item \textbf{Temperatura de Curie} es la temperatura por encima de la cual un cuerpo ferromagnético pierde su magnetismo, comportándose como un material puramente paramagnético. Esta temperatura característica lleva el nombre del físico francés Pierre Curie, que la descubrió en 1895.
    \item \textbf{Spin}: Propiedad física de las partículas elementales por el cual tienen un momento angular intrínseco de valor fijo. Las partículas con espín presentan un momento magnético, recordando a un cuerpo cargado eléctricamente en rotación, en general el ferromagnetismo que poseen algunos elementos, surge del alineamiento de los espines.
\end{itemize}

\section{Procedimiento}
\begin{enumerate}
    \item Primero,llenamos el vaso precipitado con 400 ml de agua, luego colocamos el imán en su interior.
    \item Ahora con ayuda del mechero calentamos el envase precipitado junto con el  imán, hasta alcanzar una temperatura de 10°C
    \item Con guantes y mucho cuidado alejamos el vaso precipitado del mechero, posteriormente esperamos 2 minutos para que el imán y el agua alcancen un equilibrio térmico.
    \item Con ayuda de unas pinzas, sacamos el imán del interior del vaso precipitado y calculamos su campo magnético utilizando la aplicación pyphox, a una distancia de aproximadamente 3 [cm] del celular. (repetimos pasos 2,3,4 para temperatura 20°,30°,40°,50°,60°,70°,80°,90° y 100°)
    \item De manera paralela, colocamos un imán idéntico al anterior, en el interior de un recipiente lleno de agua y lo metemos adentro de una nevera, hasta que alcance una temperatura cercana a 0°.(repetimos para una temperatura cercana a 10°)
    \item Sacamos el imán  del recipiente ,enfriado y calculamos su campo magnetico con pyphox.
\end{enumerate}
\section{Materiales}
\begin{itemize}
    \item Imán de neodimio (X2)
    \item Vaso precipitado (1000 [ml])
    \item Agua
    \item Termómetro (Termocupla)
    \item Mechero
    %\item Soporte universal
    \item Pinzas
    \item Regla metálica
    \item Magnetómetro (En Phyphox)
\end{itemize}

\section{Resultados}
Para la captura de datos la herramienta phyphox registró la intensidad del campo magnético que emanaba del imán de neodimio inmediatamente después de ser extraído de la fuente de calor, en sí la aplicación arroja un gráfico de la variación del campo magnético del elemento en el tiempo, sin embargo, nuestro interés aborda valores específicos de su intensidad, por ello solo se considera el valor del campo a un tiempo de 5 segundos.\\
Como para cada temperatura teníamos más de 1000 valores de campo magnético, tomamos como dato su promedio y su error la desviación estándar de los datos.\\
Como los 10 valores de temperatura se midieron con una termocupla de sensibilidad $0.1 [^\circ C]$, el error para cada temperatura es de $0.05 [^\circ C]$.\\
Luego se ajustaron los valores de campo magnético con respecto a su error dado con una cifra significativa y los valores de temperatura con respecto a su error con una cifra significativa.\\

\begin{table}[h!]
    \centering
    \begin{tabular}{|c|c|}
        \hline
        Temperatura [$^\circ C$] & Intensidad de campo Magnético [$\upmu T$]\\
        \hline
        $5.30 \pm 0.05$    &  $ 1713.9 \pm 0.7$ \\
        \hline
        $13.30 \pm 0.05$  &   $1669 \pm 4$ \\
        \hline
        $26.60 \pm 0.05$  &   $1487.2 \pm 0.7$\\
        \hline
        $36.70 \pm 0.05 $ &   $1366 \pm 1$\\
        \hline
        $46.50  \pm 0.05$ &   $1306 \pm 2$\\
        \hline
        $55.30 \pm 0.05 $ &   $1237 \pm 2$\\
        \hline
        $63.30  \pm 0.05$ &   $1212 \pm 2$\\
        \hline
        $79.10  \pm 0.05$ &   $833  \pm 3$\\
        \hline
        $89.20  \pm 0.05$ &   $779 \pm 1$\\
        \hline
        $100.00 \pm 0.05$ &   $711  \pm 1 $\\
        \hline

    \end{tabular}
    \caption{Intensidad de campo magnético a distintas temperaturas}
    \label{datos-tabla}
\end{table}
de los valores se deduce que a medida que la temperatura del imán aumenta, su campo magnético decrece, sin embargo no es posible indicar su grado de decaimiento, luego, al graficar estos datos se obtiene la imagen (\ref{Grafica de los datos,sin ajuste}) 

\begin{figure}[h!]
    \centering
    \includegraphics[width=9cm]{imagenes/grafico-campo-tem.pdf}
    \caption{Gráfica intensidad de campo magnético vs Temperatura}
    \label{Grafica de los datos,sin ajuste}
\end{figure}
\section{Análisis}
Utilizando los datos de la tabla (\ref{datos-tabla}), ajustaremos los datos por el método de mínimos cuadrados ponderado, ya que el error en el eje $y$ (intensidad de campo) fue calculado con la desviación estándar de un registro de datos, por lo que, el error de los puntos del campo es distinto para cada uno.    


Para trabajar con la información recopilada, usaremos las herramientas programáticas dentro de \textit{python}, en especial los módulos de \textit{Matplotlib.pyplot}, \textit{Numpy} y \textit{scipy}, en especifico, el ajuste que atraviese los datos lo determinaremos con el módulo \textit{optimize} que contiene la función \textit{Curve\_fit} capaz de obtener coeficientes en un arreglo matemático específico.

Como primera observación, se puede deducir que los datos parecen representar un decrecimiento exponencial, así que para comenzar, utilizamos el siguiente ajuste $B_0(T)=A\cdot e^{-U\cdot T}$, en donde A y U son coeficientes por determinar,  .

Como observación pertinente al modelo exponencial, se debe afirmar que es un modelo correcto para temperaturas que estén entre 0[°C] y 100[°C], no es aplicable este modelo fuera de ese intervalo de temperaturas, puesto que los registros hechos no contienen una mensuración del campo para temperaturas muy bajas (inferiores a 0) ni tampoco es posible deducir el ajuste que se hará para valores muy grandes.
Esto ya que en el caso de la exponencial que modela el ajuste, las matemáticas sugieren que al imán de neodimio se le deba suministrar una temperatura infinita para que recién el imán pierda su capacidad magnética ($B=0$) lo que es absurdo puesto que transgrede la definición de la temperatura de Curie \cite{ref4} que sabemos que para un material ferromagnético, en específico, para los imanes de neodimio, su capacidad magnética desaparece a temperaturas $T_c\in [310,400] (^\circ C)$ \cite{ref5}.


Determinaremos el ajuste de  mínimos cuadrados ponderados, esto ya que los valores de la intensidad del campo medido tienen cada uno un error individual, haciendo necesario añadir un parámetro que establezca el peso del error de cada dato, para ello aplicamos \textit{Curve\_fit} por su definición, obteniendo los valores óptimos para cuando se desea ajustar el arreglo exponencial (\ref{Ajuste_exponencial})


\begin{table}[!h]
    \centering
    \begin{tabular}{|c|c|}
      \hline
      coeficientes      & Valores \\
      \hline
      A       &   1875.8 $\pm$ 0.7 $[\mu T]$\\
      \hline
      U  & 0.008844 $\pm$ 0.000009 $[1/^\circ C]$\\  
      \hline
    
    \end{tabular}
    \caption{Coeficientes para ajuste $B(T)=A \cdot e^{-U \cdot T} $}
    \label{}
\end{table}

Al otorgar dichos coeficientes a la función, se aprecia que nuestra curva cruza por los datos de forma certera, tal cual se muestra en (\ref{Ajuste_exponencial}).




\begin{figure}
    \centering 
    \includegraphics[width=15cm]{imagenes/Ajuste_exponencial.pdf}
    \caption{Ajuste de curva para el caso de $B_0(T)=A \cdot e^{-U \cdot T} $}
    \label{Ajuste_exponencial}
\end{figure}

Notar que este ajuste exponencial tiene sentido físico si su análisis dimensional es correcto, es decir, sabemos que el campo magnético cumple que $[B]=[\mu T]$  por lo que si ajustamos los parámetros que obtuvimos con magnitudes del sistema internacional, se debería cumplir que  $[U]=[1/{^\circ C}]$ y $[A]=[\mu T]$.





%\begin{align}\label{ecuacion exponencial}
%    B(T)=A \cdot e^{-U \cdot T}   ,\\
%    
%\end{align}




Por otro lado, otro posible ajuste que represente la situación, es considerar una recta, de la forma $B_1(T)=AT+B$, esto claramente es un ajuste adecuado por la misma razón que se ha ajustado una exponencial (valores decrecientes paulatinamente).

En cuanto a la recta, también presentaría inconsistencias para datos de campo menores a $0 [^\circ C]$ y mayores a $100 [^\circ C]$, el ajuste supondría un incremento muy grande de intensidad magnética cuando las temperaturas son muy inferiores a $0[^\circ C]$ lo que no tiene tanto sentido ya que su capacidad magnética esta limitada al tamaño del imán (a la cantidad de momentos magnéticos internos).


Aplicando nuevamente la herramienta de optimización \emph{polyfit} del  módulo \emph{numpy} a  la  nueva  función junto con los datos, aplicamos el método de mínimos cuadrados ponderado, ya que los errores en el eje Y (campo) son variables, de estos se obtienen los siguientes valores de los coeficientes, que además se encuentran ajustados con sus respectivos errores.

Ajustamos la curva con los siguientes parámetros
\begin{table}[!h]
    \centering
    \begin{tabular}{|c|c|}
      \hline
      coeficientes      & Valores \\
      \hline
      A       &   -11.1 $\pm$ 0.6 $[\mu T/^\circ C]$\\
      \hline
      U  & 1801.51 $\pm $ 0.01 $[\mu T]$\\
      \hline
    
    \end{tabular}
    \caption{Coeficientes para ajuste $B_1(T)=AT+U $}
    \label{param-recta}
\end{table}
este ajuste de los datos se aprecian en la figura (\ref{Ajuste-recta}).


\begin{figure}[h!]
    \centering
    \includegraphics[width=7cm]{imagenes/Ajuste_recta.pdf}
    \caption{Ajuste de curva para el caso de $B_1(T)=A T+U $}
    \label{Ajuste-recta}
\end{figure}


Notar que esta igualdad tiene sentido físico si su análisis dimensional es correcto, es decir, sabemos que el campo magnético cumple que $[B]=[\mu T]$  por lo que si ajustamos los parámetros que obtuvimos con magnitudes del sistema internacional, se debería cumplir que  $[U]=[\mu T]$ y $[A]=[\mu T/^\circ C]$.

En este punto, tenemos dos posibles ajustes que modelan este fenómeno, sin embargo, a pesar de que ambos pueden otorgar valores muy cercanos a la realidad, debe ocurrir que uno sea más preciso que el otro, entonces para decidir cuál es el ajuste más óptimo, procederemos a verificar que las curvas que se ajustaron hayan sido una buen elección haciendo un análisis de los residuos, usando la ecuación $\epsilon_i= y_i - f(x_i)/\Delta y_i$ , que determina la distancia que hay de los datos al ajuste realizado, con $\Delta y_i$ el error de cada dato de campo magnético $y_i$, para cada valor se obtienen los siguientes arreglos en cada ajuste hecho.\\



\begin{table}[h!]
    \centering
    \begin{tabular}{|c|c|c|}
    \hline
      Coeficiente de determinación &   Recta & Exponencial \\
         \hline
          $R^2$ & 0.97& 0.95 \\
          \hline
    \end{tabular}
    \caption{Coeficiente de determinación de los ajustes}
    \label{tabla de residuos}
\end{table}










%----------------------------------------------------------------
%Graficos de residuos exponencial
%---------------------------------------------------------------


\begin{figure}[h!]
    \centering
    \includegraphics[width=7cm]{imagenes/Residuos_exponencial_normalizado.pdf}
    \caption{Gráfica de los residuos normalizados de la exponencial $B_0 (T)$}
    \label{Residuos-exponencial}
\end{figure}

%Al graficar los datos vs los residuos (\ref{Residuos-exponencial}) podemos ver que estos no siguen una tendencia clara, considerando que poseíamos pocos datos , por lo que podemos decir que hicimos un buen ajuste.\\  


%----------------------------------------------------------------
%Graficos de residuos recta
%---------------------------------------------------------------
\begin{figure}[h!]
    \centering
    \includegraphics[width=7cm]{imagenes/Residuos_recta_normalizados.pdf}
    \caption{Gráfica de los residuos normalizados para la recta $B_1(T)$}
    \label{Residuos_recta}
\end{figure}

Del gráfico de los residuos para cada función se deduce que ambos ajustes son efectivos, ya que los coeficientes de determinación, tanto para la exponencial como la recta son valores cercanos a $R^2=1$, sin embargo, al comparar ambos ajustes, claramente la recta posee una acomodación más cercana a los datos, siendo un ajuste más preciso, que a simple vista se puede apreciar en la distribución más aleatoria de los puntos en el gráfico de los residuos de la recta (\ref{Residuos_recta}) en comparación a los puntos de los residuos de la exponencial (\ref{Residuos-exponencial}) .









 
\section{Conclusión}

Al finalizar este experimento pudimos darnos cuenta de que la temperatura que es inducida a un imán (temperaturas altas y bajas) influye en la intensidad del campo magnético. Esto debido a que logra alterar el orden de las partículas que generan los campos o dicho de otra forma los spins, ya sea desordenándolas y con ello debilitando el campo magnético, o  provocado por las altas temperaturas, creando un espacio donde los spins pueden ordenarse de mejor manera, provocado por las bajas temperaturas.

Donde además el decrecimiento del campo magnético se puede comparar o aproximar a la forma de una exponencial inversa.

En algún momento si se volviese a realizar este experimento desearíamos tener herramientas adecuadas para lograr analizar lo que sucede con el campo magnético de un imán abarcando más mediciones, ya sea con temperaturas superiores a los $100 [^\circ C]$ y/o a temperaturas menores a los $0 [^\circ  C]$ y formar un ajuste mucho mejor y más amplio al comportamiento de un imán ante diferentes grados de temperaturas.


 
\begin{thebibliography}{}


\bibitem[1]{ref1} Alcázar, G. A. P. (2016, 3 julio). \textit{ Imanes permanentes: características, aplicaciones y futuro | Revista de la Academia Colombiana de Ciencias Exactas, Físicas y Naturales}.  from \\
\hyperlink{https://raccefyn.co/index.php/raccefyn/article/view/361}{https://raccefyn.co/index.php/raccefyn/article/view/361}

\bibitem[2]{ref2} Staacks, S. (n.d.). Your smartphone is a mobile lab. Phyphox. Retrieved October 27, 2022, from \\
\hyperlink{https://phyphox.org}{https://phyphox.org}

\bibitem[3]{ref3} Kirchmayr, H. R. (1966). Permanent magnets and hard magnetic materials. Journal of Physics D: Applied Physics. 
\hyperlink{https://iopscience.iop.org/article/10.1088/0022-3727/29/11/007}{https://iopscience.iop.org/article/10.1088/0022-3727/29/11/007}


\bibitem[4]{ref4} Fabian, K., Shcherbakov, V. P. \& McEnroe, S. A. (2013, 24 abril). Measuring the Curie temperature. AGU \-Geochemistry Geophysics Geosystems.\\
\hyperlink{https://onlinelibrary.wiley.com/resolve/doi?DOI=10.1029/2012GC004440}{https://onlinelibrary.wiley.com/resolve/doi?DOI=10.1029/2012GC004440}

\bibitem[5]{ref5}Especializados en imanes, especificación de las propiedades del neodimio. - Magfine. (s. f.). magfine. Recuperado 9 de noviembre de 2022, de \\
\hyperlink{https://www.magfine.it/es/user\_data/neodymium\_property.php}{https://www.magfine.it/es/user\_data/neodymium\_property.php}





\end{thebibliography}

\end{document}
