\section{Conclusión}

Al finalizar este experimento pudimos darnos cuenta de que la temperatura que es inducida a un imán (temperaturas altas y bajas) influye en la intensidad del campo magnético. Esto debido a que logra alterar el orden de las partículas que generan los campos o dicho de otra forma los spins, ya sea desordenándolas y con ello debilitando el campo magnético, o  provocado por las altas temperaturas, creando un espacio donde los spins pueden ordenarse de mejor manera, provocado por las bajas temperaturas.

Donde además el decrecimiento del campo magnético se puede comparar o aproximar a la forma de una exponencial inversa.

En algún momento si se volviese a realizar este experimento desearíamos tener herramientas adecuadas para lograr analizar lo que sucede con el campo magnético de un imán abarcando más mediciones, ya sea con temperaturas superiores a los $100 [^\circ C]$ y/o a temperaturas menores a los $0 [^\circ  C]$ y formar un ajuste mucho mejor y más amplio al comportamiento de un imán ante diferentes grados de temperaturas.


